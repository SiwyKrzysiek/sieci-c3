\documentclass{mwart} % Polska wersja klasy article

\usepackage{polski} % Pozwala na użycie polskiego. Ustawia między innymi fontenc na T1
\usepackage[utf8]{inputenc} % Informuje o kodowaniu
\usepackage{textcomp} % Znaki specjalne takie jak ~
\usepackage{xcolor} % Definicje kolorów

\renewcommand{\labelitemi}{\textbullet} % Zmiana symbolu wliczeń

\usepackage{graphicx}
\graphicspath{ {./Obrazy/} }
\usepackage{float} % Pozycjonowanie figur
\usepackage{mwe} % Tymczasowe grafiki

\usepackage{listings} % Listingi kodu
\lstset{basicstyle=\ttfamily,
  showstringspaces=false,
  commentstyle=\color{gray},
  keywordstyle=\color{blue}
}

\title{Laboratorium sieci komputerowych - c3 \\ Tworzenie i badanie sieci wewnętrznych}
\author{Krzysztof Dąbrowski gr. 3}
\date{\today}

\begin{document}
\maketitle{}
\tableofcontents{}
%\newpage

\section{Cel zajęć}
Celem laboratoriów \textit{c3} było utworzenie kilku sieci wewnętrznych oraz podłączenie do nich interfejsów maszyn wirtualnych. W celu nadania adresów wykorzystane zostało adresowanie statyczne oraz dynamiczne. Po zakończeniu konfiguracji sieci należało przeprowadzić analizę ruchu sieciowego.

\section{Schemat sieci}
Do wykonania zadań została utworzona sieć o schemacie przedstawionym poniżej.

%TODO: Schemat sieci
\begin{figure}[H]
  \centering
  \includegraphics[width=\textwidth]{example-image-a}
  
  \caption{Schemat budowanej sieci}
  \label{fig:SchematSieci}
\end{figure}

\section{Statyczne adresowanie}
%TODO: Statyczne adresowanie
%TODO: Spytać tatę czy może być puste
Ręcznie wybiorę adresy, które przypiszę statycznie interfejsom maszyn.

\subsection{Wybór adresów}
Ponieważ wiem, że będę potrzebował 2 sieci postanowiłem podzielić prywatną sieć \texttt{192.168.0.0} na dwie podsieci. W celu ułatwienia obliczeń postanowiłem, że maska podsieci będzie \textbf{24 bitowa}.

\begin{itemize}
  \item Adres pierwszej sieci -- \texttt{192.168.0.0/24}.
  \item Adres drugiej sieci -- \texttt{192.168.1.0/24}.
\end{itemize}

Maszyna Vm1 otrzyma statyczny adres \texttt{192.168.0.1/24}, a maszyna Vm2 \texttt{192.168.0.2/24}.


\section{Dynamiczne adresowanie}
%TODO: Dynamiczne adresowanie

\section{Druga warstwa sieciowa}
%TODO: Druga warstwa sieciowa

\section{Analiza ruchu sieciowego}
%TODO: Analiza ruchu sieciowego

\end{document}
